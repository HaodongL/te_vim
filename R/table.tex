% Options for packages loaded elsewhere
\PassOptionsToPackage{unicode}{hyperref}
\PassOptionsToPackage{hyphens}{url}
%
\documentclass[
]{article}
\usepackage{amsmath,amssymb}
\usepackage{lmodern}
\usepackage{iftex}
\ifPDFTeX
  \usepackage[T1]{fontenc}
  \usepackage[utf8]{inputenc}
  \usepackage{textcomp} % provide euro and other symbols
\else % if luatex or xetex
  \usepackage{unicode-math}
  \defaultfontfeatures{Scale=MatchLowercase}
  \defaultfontfeatures[\rmfamily]{Ligatures=TeX,Scale=1}
\fi
% Use upquote if available, for straight quotes in verbatim environments
\IfFileExists{upquote.sty}{\usepackage{upquote}}{}
\IfFileExists{microtype.sty}{% use microtype if available
  \usepackage[]{microtype}
  \UseMicrotypeSet[protrusion]{basicmath} % disable protrusion for tt fonts
}{}
\makeatletter
\@ifundefined{KOMAClassName}{% if non-KOMA class
  \IfFileExists{parskip.sty}{%
    \usepackage{parskip}
  }{% else
    \setlength{\parindent}{0pt}
    \setlength{\parskip}{6pt plus 2pt minus 1pt}}
}{% if KOMA class
  \KOMAoptions{parskip=half}}
\makeatother
\usepackage{xcolor}
\usepackage[margin=1in]{geometry}
\usepackage{color}
\usepackage{fancyvrb}
\newcommand{\VerbBar}{|}
\newcommand{\VERB}{\Verb[commandchars=\\\{\}]}
\DefineVerbatimEnvironment{Highlighting}{Verbatim}{commandchars=\\\{\}}
% Add ',fontsize=\small' for more characters per line
\usepackage{framed}
\definecolor{shadecolor}{RGB}{248,248,248}
\newenvironment{Shaded}{\begin{snugshade}}{\end{snugshade}}
\newcommand{\AlertTok}[1]{\textcolor[rgb]{0.94,0.16,0.16}{#1}}
\newcommand{\AnnotationTok}[1]{\textcolor[rgb]{0.56,0.35,0.01}{\textbf{\textit{#1}}}}
\newcommand{\AttributeTok}[1]{\textcolor[rgb]{0.77,0.63,0.00}{#1}}
\newcommand{\BaseNTok}[1]{\textcolor[rgb]{0.00,0.00,0.81}{#1}}
\newcommand{\BuiltInTok}[1]{#1}
\newcommand{\CharTok}[1]{\textcolor[rgb]{0.31,0.60,0.02}{#1}}
\newcommand{\CommentTok}[1]{\textcolor[rgb]{0.56,0.35,0.01}{\textit{#1}}}
\newcommand{\CommentVarTok}[1]{\textcolor[rgb]{0.56,0.35,0.01}{\textbf{\textit{#1}}}}
\newcommand{\ConstantTok}[1]{\textcolor[rgb]{0.00,0.00,0.00}{#1}}
\newcommand{\ControlFlowTok}[1]{\textcolor[rgb]{0.13,0.29,0.53}{\textbf{#1}}}
\newcommand{\DataTypeTok}[1]{\textcolor[rgb]{0.13,0.29,0.53}{#1}}
\newcommand{\DecValTok}[1]{\textcolor[rgb]{0.00,0.00,0.81}{#1}}
\newcommand{\DocumentationTok}[1]{\textcolor[rgb]{0.56,0.35,0.01}{\textbf{\textit{#1}}}}
\newcommand{\ErrorTok}[1]{\textcolor[rgb]{0.64,0.00,0.00}{\textbf{#1}}}
\newcommand{\ExtensionTok}[1]{#1}
\newcommand{\FloatTok}[1]{\textcolor[rgb]{0.00,0.00,0.81}{#1}}
\newcommand{\FunctionTok}[1]{\textcolor[rgb]{0.00,0.00,0.00}{#1}}
\newcommand{\ImportTok}[1]{#1}
\newcommand{\InformationTok}[1]{\textcolor[rgb]{0.56,0.35,0.01}{\textbf{\textit{#1}}}}
\newcommand{\KeywordTok}[1]{\textcolor[rgb]{0.13,0.29,0.53}{\textbf{#1}}}
\newcommand{\NormalTok}[1]{#1}
\newcommand{\OperatorTok}[1]{\textcolor[rgb]{0.81,0.36,0.00}{\textbf{#1}}}
\newcommand{\OtherTok}[1]{\textcolor[rgb]{0.56,0.35,0.01}{#1}}
\newcommand{\PreprocessorTok}[1]{\textcolor[rgb]{0.56,0.35,0.01}{\textit{#1}}}
\newcommand{\RegionMarkerTok}[1]{#1}
\newcommand{\SpecialCharTok}[1]{\textcolor[rgb]{0.00,0.00,0.00}{#1}}
\newcommand{\SpecialStringTok}[1]{\textcolor[rgb]{0.31,0.60,0.02}{#1}}
\newcommand{\StringTok}[1]{\textcolor[rgb]{0.31,0.60,0.02}{#1}}
\newcommand{\VariableTok}[1]{\textcolor[rgb]{0.00,0.00,0.00}{#1}}
\newcommand{\VerbatimStringTok}[1]{\textcolor[rgb]{0.31,0.60,0.02}{#1}}
\newcommand{\WarningTok}[1]{\textcolor[rgb]{0.56,0.35,0.01}{\textbf{\textit{#1}}}}
\usepackage{graphicx}
\makeatletter
\def\maxwidth{\ifdim\Gin@nat@width>\linewidth\linewidth\else\Gin@nat@width\fi}
\def\maxheight{\ifdim\Gin@nat@height>\textheight\textheight\else\Gin@nat@height\fi}
\makeatother
% Scale images if necessary, so that they will not overflow the page
% margins by default, and it is still possible to overwrite the defaults
% using explicit options in \includegraphics[width, height, ...]{}
\setkeys{Gin}{width=\maxwidth,height=\maxheight,keepaspectratio}
% Set default figure placement to htbp
\makeatletter
\def\fps@figure{htbp}
\makeatother
\setlength{\emergencystretch}{3em} % prevent overfull lines
\providecommand{\tightlist}{%
  \setlength{\itemsep}{0pt}\setlength{\parskip}{0pt}}
\setcounter{secnumdepth}{-\maxdimen} % remove section numbering
\usepackage{booktabs}
\usepackage{longtable}
\usepackage{array}
\usepackage{multirow}
\usepackage{wrapfig}
\usepackage{float}
\usepackage{colortbl}
\usepackage{pdflscape}
\usepackage{tabu}
\usepackage{threeparttable}
\usepackage{threeparttablex}
\usepackage[normalem]{ulem}
\usepackage{makecell}
\usepackage{xcolor}
\ifLuaTeX
  \usepackage{selnolig}  % disable illegal ligatures
\fi
\IfFileExists{bookmark.sty}{\usepackage{bookmark}}{\usepackage{hyperref}}
\IfFileExists{xurl.sty}{\usepackage{xurl}}{} % add URL line breaks if available
\urlstyle{same} % disable monospaced font for URLs
\hypersetup{
  pdftitle={Untitled},
  pdfauthor={Haodong Li},
  hidelinks,
  pdfcreator={LaTeX via pandoc}}

\title{Untitled}
\author{Haodong Li}
\date{2022-10-03}

\begin{document}
\maketitle

\begin{verbatim}
## 
## Please cite as:
\end{verbatim}

\begin{verbatim}
##  Hlavac, Marek (2022). stargazer: Well-Formatted Regression and Summary Statistics Tables.
\end{verbatim}

\begin{verbatim}
##  R package version 5.2.3. https://CRAN.R-project.org/package=stargazer
\end{verbatim}

\begin{verbatim}
## 
## Attaching package: 'dplyr'
\end{verbatim}

\begin{verbatim}
## The following object is masked from 'package:kableExtra':
## 
##     group_rows
\end{verbatim}

\begin{verbatim}
## The following objects are masked from 'package:stats':
## 
##     filter, lag
\end{verbatim}

\begin{verbatim}
## The following objects are masked from 'package:base':
## 
##     intersect, setdiff, setequal, union
\end{verbatim}

\begin{verbatim}
## 
## Attaching package: 'data.table'
\end{verbatim}

\begin{verbatim}
## The following objects are masked from 'package:dplyr':
## 
##     between, first, last
\end{verbatim}

\begin{Shaded}
\begin{Highlighting}[]
\NormalTok{output\_filename }\OtherTok{\textless{}{-}} \FunctionTok{paste0}\NormalTok{(}\StringTok{\textquotesingle{}\textasciitilde{}/Repo/te\_vim/simu\_res/theta\_s/\textquotesingle{}}\NormalTok{,}\StringTok{"local\_"}\NormalTok{, }\DecValTok{500}\NormalTok{, }\StringTok{"\_"}\NormalTok{, }\StringTok{\textquotesingle{}2022{-}10{-}02\textquotesingle{}}\NormalTok{,}\StringTok{\textquotesingle{}.csv\textquotesingle{}}\NormalTok{)}
\NormalTok{res1 }\OtherTok{\textless{}{-}} \FunctionTok{read\_csv}\NormalTok{(output\_filename) }\SpecialCharTok{\%\textgreater{}\%} \FunctionTok{mutate}\NormalTok{(}\AttributeTok{n =} \DecValTok{500}\NormalTok{)}
\end{Highlighting}
\end{Shaded}

\begin{verbatim}
## New names:
## Rows: 500 Columns: 11
## -- Column specification
## -------------------------------------------------------- Delimiter: "," chr
## (1): ...1 dbl (10): i, truth, cvtmle, cvtmle_se, cvtmle_lower, cvtmle_upper,
## cvaiptw, ...
## i Use `spec()` to retrieve the full column specification for this data. i
## Specify the column types or set `show_col_types = FALSE` to quiet this message.
## * `` -> `...1`
\end{verbatim}

\begin{Shaded}
\begin{Highlighting}[]
\NormalTok{output\_filename }\OtherTok{\textless{}{-}} \FunctionTok{paste0}\NormalTok{(}\StringTok{\textquotesingle{}\textasciitilde{}/Repo/te\_vim/simu\_res/theta\_s/\textquotesingle{}}\NormalTok{,}\StringTok{"local\_"}\NormalTok{, }\DecValTok{2000}\NormalTok{, }\StringTok{"\_"}\NormalTok{, }\StringTok{\textquotesingle{}2022{-}10{-}02\textquotesingle{}}\NormalTok{,}\StringTok{\textquotesingle{}.csv\textquotesingle{}}\NormalTok{)}
\NormalTok{res2 }\OtherTok{\textless{}{-}} \FunctionTok{read\_csv}\NormalTok{(output\_filename) }\SpecialCharTok{\%\textgreater{}\%} \FunctionTok{mutate}\NormalTok{(}\AttributeTok{n =} \DecValTok{2000}\NormalTok{)}
\end{Highlighting}
\end{Shaded}

\begin{verbatim}
## New names:
## Rows: 500 Columns: 11
## -- Column specification
## -------------------------------------------------------- Delimiter: "," chr
## (1): ...1 dbl (10): i, truth, cvtmle, cvtmle_se, cvtmle_lower, cvtmle_upper,
## cvaiptw, ...
## i Use `spec()` to retrieve the full column specification for this data. i
## Specify the column types or set `show_col_types = FALSE` to quiet this message.
## * `` -> `...1`
\end{verbatim}

\begin{Shaded}
\begin{Highlighting}[]
\NormalTok{output\_filename }\OtherTok{\textless{}{-}} \FunctionTok{paste0}\NormalTok{(}\StringTok{\textquotesingle{}\textasciitilde{}/Repo/te\_vim/simu\_res/theta\_s/\textquotesingle{}}\NormalTok{,}\StringTok{"local\_"}\NormalTok{, }\DecValTok{5000}\NormalTok{, }\StringTok{"\_"}\NormalTok{, }\StringTok{\textquotesingle{}2022{-}10{-}02\textquotesingle{}}\NormalTok{,}\StringTok{\textquotesingle{}.csv\textquotesingle{}}\NormalTok{)}
\NormalTok{res3 }\OtherTok{\textless{}{-}} \FunctionTok{read\_csv}\NormalTok{(output\_filename) }\SpecialCharTok{\%\textgreater{}\%} \FunctionTok{mutate}\NormalTok{(}\AttributeTok{n =} \DecValTok{5000}\NormalTok{)}
\end{Highlighting}
\end{Shaded}

\begin{verbatim}
## New names:
## Rows: 500 Columns: 11
## -- Column specification
## -------------------------------------------------------- Delimiter: "," chr
## (1): ...1 dbl (10): i, truth, cvtmle, cvtmle_se, cvtmle_lower, cvtmle_upper,
## cvaiptw, ...
## i Use `spec()` to retrieve the full column specification for this data. i
## Specify the column types or set `show_col_types = FALSE` to quiet this message.
## * `` -> `...1`
\end{verbatim}

\begin{Shaded}
\begin{Highlighting}[]
\NormalTok{output\_filename }\OtherTok{\textless{}{-}} \FunctionTok{paste0}\NormalTok{(}\StringTok{\textquotesingle{}\textasciitilde{}/Repo/te\_vim/simu\_res/theta\_s/\textquotesingle{}}\NormalTok{,}\StringTok{"local\_"}\NormalTok{, }\DecValTok{7000}\NormalTok{, }\StringTok{"\_"}\NormalTok{, }\StringTok{\textquotesingle{}2022{-}10{-}03\textquotesingle{}}\NormalTok{,}\StringTok{\textquotesingle{}.csv\textquotesingle{}}\NormalTok{)}
\NormalTok{res4 }\OtherTok{\textless{}{-}} \FunctionTok{read\_csv}\NormalTok{(output\_filename) }\SpecialCharTok{\%\textgreater{}\%} \FunctionTok{mutate}\NormalTok{(}\AttributeTok{n =} \DecValTok{7000}\NormalTok{)}
\end{Highlighting}
\end{Shaded}

\begin{verbatim}
## New names:
## Rows: 500 Columns: 11
## -- Column specification
## -------------------------------------------------------- Delimiter: "," chr
## (1): ...1 dbl (10): i, truth, cvtmle, cvtmle_se, cvtmle_lower, cvtmle_upper,
## cvaiptw, ...
## i Use `spec()` to retrieve the full column specification for this data. i
## Specify the column types or set `show_col_types = FALSE` to quiet this message.
## * `` -> `...1`
\end{verbatim}

\begin{Shaded}
\begin{Highlighting}[]
\NormalTok{output\_filename }\OtherTok{\textless{}{-}} \FunctionTok{paste0}\NormalTok{(}\StringTok{\textquotesingle{}\textasciitilde{}/Repo/te\_vim/simu\_res/theta\_s/\textquotesingle{}}\NormalTok{,}\StringTok{"local\_"}\NormalTok{, }\DecValTok{10000}\NormalTok{, }\StringTok{"\_"}\NormalTok{, }\StringTok{\textquotesingle{}2022{-}10{-}03\textquotesingle{}}\NormalTok{,}\StringTok{\textquotesingle{}.csv\textquotesingle{}}\NormalTok{)}
\NormalTok{res5 }\OtherTok{\textless{}{-}} \FunctionTok{read\_csv}\NormalTok{(output\_filename) }\SpecialCharTok{\%\textgreater{}\%} \FunctionTok{mutate}\NormalTok{(}\AttributeTok{n =} \DecValTok{10000}\NormalTok{)}
\end{Highlighting}
\end{Shaded}

\begin{verbatim}
## New names:
## Rows: 500 Columns: 11
## -- Column specification
## -------------------------------------------------------- Delimiter: "," chr
## (1): ...1 dbl (10): i, truth, cvtmle, cvtmle_se, cvtmle_lower, cvtmle_upper,
## cvaiptw, ...
## i Use `spec()` to retrieve the full column specification for this data. i
## Specify the column types or set `show_col_types = FALSE` to quiet this message.
## * `` -> `...1`
\end{verbatim}

\begin{Shaded}
\begin{Highlighting}[]
\NormalTok{res }\OtherTok{\textless{}{-}} \FunctionTok{rbind}\NormalTok{(res1, res2, res3, res4, res5)}
\end{Highlighting}
\end{Shaded}

\begin{verbatim}
## `summarise()` has grouped output by 'n'. You can override using the `.groups`
## argument.
\end{verbatim}

\#wide to long

\#merge

\newpage

\begin{Shaded}
\begin{Highlighting}[]
\NormalTok{data\_long }\SpecialCharTok{\%\textgreater{}\%} 
  \FunctionTok{mutate}\NormalTok{(}\FunctionTok{across}\NormalTok{(}\FunctionTok{where}\NormalTok{(is.numeric), }\SpecialCharTok{\textasciitilde{}} \FunctionTok{round}\NormalTok{(., }\DecValTok{3}\NormalTok{))) }\SpecialCharTok{\%\textgreater{}\%} 
  \FunctionTok{kable}\NormalTok{(}\StringTok{"latex"}\NormalTok{, }\AttributeTok{booktabs =}\NormalTok{ T, }\AttributeTok{caption =} \StringTok{"Performance of CV{-}TMLE and CV{-}EE for Theta"}\NormalTok{) }\SpecialCharTok{\%\textgreater{}\%} 
  \FunctionTok{collapse\_rows}\NormalTok{(}\AttributeTok{columns =} \DecValTok{1}\NormalTok{, }\AttributeTok{latex\_hline =} \StringTok{"major"}\NormalTok{, }\AttributeTok{valign =} \StringTok{"middle"}\NormalTok{)}\SpecialCharTok{\%\textgreater{}\%}
  \FunctionTok{kable\_styling}\NormalTok{(}\AttributeTok{latex\_options =} \StringTok{"scale\_down"}\NormalTok{)}
\end{Highlighting}
\end{Shaded}

\begin{table}

\caption{\label{tab:unnamed-chunk-6}Performance of CV-TMLE and CV-EE for Theta}
\centering
\resizebox{\linewidth}{!}{
\begin{tabular}[t]{rlrrrrrrr}
\toprule
n & Method & True\_Theta & Variance & Bias & MSE & Coverage & Coverage\_or & CI\_width\\
\midrule
 & CV-TMLE & 0.686 & 0.027 & -0.132 & 0.044 & 0.758 & 0.866 & 0.568\\

\multirow{-2}{*}{\raggedleft\arraybackslash 500} & CV-EE & 0.686 & 0.029 & -0.173 & 0.059 & 0.734 & 0.834 & 0.614\\
\cmidrule{1-9}
 & CV-TMLE & 0.686 & 0.007 & -0.051 & 0.009 & 0.830 & 0.890 & 0.295\\

\multirow{-2}{*}{\raggedleft\arraybackslash 2000} & CV-EE & 0.686 & 0.006 & -0.059 & 0.010 & 0.840 & 0.882 & 0.301\\
\cmidrule{1-9}
 & CV-TMLE & 0.686 & 0.003 & -0.026 & 0.003 & 0.894 & 0.920 & 0.190\\

\multirow{-2}{*}{\raggedleft\arraybackslash 5000} & CV-EE & 0.686 & 0.003 & -0.028 & 0.003 & 0.896 & 0.916 & 0.190\\
\cmidrule{1-9}
 & CV-TMLE & 0.686 & 0.002 & -0.020 & 0.002 & 0.888 & 0.920 & 0.161\\

\multirow{-2}{*}{\raggedleft\arraybackslash 7000} & CV-EE & 0.686 & 0.002 & -0.020 & 0.002 & 0.888 & 0.916 & 0.161\\
\cmidrule{1-9}
 & CV-TMLE & 0.686 & 0.001 & -0.021 & 0.002 & 0.906 & 0.906 & 0.134\\

\multirow{-2}{*}{\raggedleft\arraybackslash 10000} & CV-EE & 0.686 & 0.001 & -0.020 & 0.001 & 0.902 & 0.900 & 0.134\\
\bottomrule
\end{tabular}}
\end{table}

\end{document}
